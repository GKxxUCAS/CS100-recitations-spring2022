\documentclass{beamer}

\usepackage{amsmath}
\usepackage{amsthm}
\usepackage{float}
\usepackage{graphicx}
\usepackage{color}
\usepackage{url}

\usepackage{../cppenv}

\usepackage{../recdefs}

\usetheme{Antibes}

\title{CS100 Recitation 6}
\subtitle{Dynamically Expanding Storage}
\author{GKxx}
\date{March 28, 2022}

\begin{document}

\begin{frame}
    \maketitle
\end{frame}

\begin{frame}{Motivation}
    \begin{itemize}
        \item Store a \blue{compile-time-determined} amount of data
        \item Store a \blue{runtime-determined} amount of data
        \item Store an \blue{unknown} amount of data
    \end{itemize}
\end{frame}

\begin{frame}{Motivation}
    \begin{itemize}
        \item Store a \blue{compile-time-determined} amount of data:\\
        \red{Built-in arrays}.
        \item Store a \blue{runtime-determined} amount of data:\\
        \red{Allocate memory on heap (\ttt{malloc}, \ttt{free}, etc.)}.
        \item Store an \blue{unknown} amount of data?
        \pause
        \begin{itemize}
            \item Suppose we want to create a \blue{list} by appending \(n\) elements one-by-one, as in \ttt{Python}...
            \pause
            \item We need some kind of storage that can \red{dynamically grow}.
        \end{itemize}
    \end{itemize}
\end{frame}

\begin{frame}{What can we do?}
    \begin{itemize}
        \item We can allocate a specific number of bytes of memory on heap.
        \item We \textbf{cannot} specify the \blue{exact location} of the memory allocated. \red{(Why?)}
    \end{itemize}
\end{frame}

\begin{frame}{A Basic Idea}
    Suppose we have stored \(n\) elements in some \blue{contiguous} memory \ttt{p[0]}, \dots, \ttt{p[n-1]}. When the \((n+1)\)-th element \(x\) comes...
    \begin{itemize}
        \item We cannot force the system to allocate the space at \ttt{p[n]}.
        \pause
        \item Naive idea:
        \begin{enumerate}
            \item Allocate another block of memory \ttt{q[0]}, \dots, \ttt{q[n]} that can contain \(n+1\) elements.
            \item Copy the original \(n\) elements to the new place.
            \item Place \(x\) at \ttt{q[n]}.
            \pause
            \item Are we done?
        \end{enumerate}
    \end{itemize}
\end{frame}

\begin{frame}{A Basic Idea}
    Suppose we have stored \(n\) elements in some \blue{contiguous} memory \ttt{p[0]}, \dots, \ttt{p[n-1]} \textbf{(dynamically allocated)}. When the \((n+1)\)-th element \(x\) comes...
    \begin{itemize}
        \item We cannot force the system to allocate the space at \ttt{p[n]}.
        \item Naive idea:
        \begin{enumerate}
            \item Allocate another block of memory \ttt{q[0]}, \dots, \ttt{q[n]} that can contain \(n+1\) elements.
            \item Copy the original \(n\) elements to the new place.
            \item Place \(x\) at \ttt{q[n]}.
            \item \red{\ttt{free(p)}!}
        \end{enumerate}
    \end{itemize}
\end{frame}

\begin{frame}[fragile]{A Basic Idea}
    \begin{cpp}
int *new_data
    = (int *)malloc(sizeof(int) * (n + 1));
for (size_t i = 0; i < n; ++i)
  new_data[i] = data[i];
new_data[n] = x;
free(data);
data = new_data;
    \end{cpp}
    \pause
    \begin{question}
        How many times of copying will happen if we append \(n\) elements one-by-one?
    \end{question}
\end{frame}

\begin{frame}{Reduce Copying}
    The number of times of copying that will happen is
    \[\sum_{i=1}^n(i-1)=\frac{n(n-1)}2,\]
    which is \red{quadratic} in \(n\). \gray{(Time complexity: \(O(n^2)\))}
    \pause
    \begin{itemize}
        \item What if we allocate more space each time?
        \pause
        \item If we allocate space for \red{\(2n\) elements}, we don't need to copy anything when appending the \((n+1)\)-th, \((n+2)\)-th, \dots, \(2n\)-th elements.
        \pause
        \begin{itemize}
            \item \(2n\) and \(n\) are not so different for computers. Don't worry!
        \end{itemize}
    \end{itemize}
\end{frame}

\begin{frame}{A Better Way}
    If we append \blue{\(n=2^m\) elements} one-by-one, the number of times of copying is
    \[\sum_{i=0}^{m-1}2^i=2^m-1=n-1,\]
    which is \red{linear} in \(n\).
    \begin{itemize}
        \item This idea is adopted in the \blue{C++ \ttt{vector}} library.
    \end{itemize}
    \pause
    \begin{question}
        Can we do better than linear time?
    \end{question}
\end{frame}

\begin{frame}{Another Idea}
    \begin{itemize}
        \item What if we don't store data in contiguous memory?
        \pause
        \item Suppose we have an element \(x\) stored somewhere.
        \item When another element \(y\) comes, just allocate the memory for \(y\), but let \(x\) \textit{somehow} \textbf{record} the location of \(y\).
    \end{itemize}
\end{frame}

\begin{frame}[fragile]{Linked-lists}
    \begin{cpp}
typedef struct _record_ {
  int data;
  struct _record_ *next_loc;
} Recorded_data;
    \end{cpp}
    \pause
    Such data structure formed by linking the elements one after another is called the \blue{linked-list}.
\end{frame}

\begin{frame}[fragile]{Linked-lists}
    Pros and cons?
    \pause
    \begin{itemize}
        \item Linked-lists are not only \blue{dynamically growing}, but also allowing elements to be inserted/removed anywhere.
        \begin{itemize}
            \item In contiguous memory, you need to move all the elements afterwards if you want to insert or remove something in the middle.
        \end{itemize}
        \pause
        \item However, random-access of data is not supported.
        \pause
        \item Need some changes to allow \blue{reverse traversal} (e.g. Doubly-linking).
    \end{itemize}
    \pause
    You will learn more in CS101: Algorithm and Data Structures.
\end{frame}

\begin{frame}{In the End...}
    \begin{itemize}
        \item What if the \textbf{type} of data to be stored is unknown?
        \item How can we store different types of data in one list?
        \item The functions `\ttt{create}' and `\ttt{destroy}' should be called by the user. How can we make them run automatically?
        \item Assignment and comparison need special named-functions. Can we use \textbf{built-in operators} naturally?
        \item How can we handle potential \textbf{errors}, like running out of memory or accessing invalid position?
    \end{itemize}
    \pause
    \rightline{\blue{Enter the C++ world to find the answers!}}
\end{frame}

\end{document}