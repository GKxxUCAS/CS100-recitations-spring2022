\documentclass{beamer}

\usepackage{../cppenv}
\usepackage{../recdefs}

\usepackage{float, graphicx}

\usetheme{EastLansing}

\title{CS100 Recitation 7}
\author{GKxx}
\date{April 4, 2022}

\AtBeginSection{
    \begin{frame}{Content}
        \tableofcontents[currentsection]
    \end{frame}
}

\begin{document}

\begin{frame}
    \maketitle
\end{frame}

\section{Idea of Encapsulation}

\begin{frame}{Drawbacks of a Simple \struct}
    Take the \ttt{Linked\_list} as an example:
    \begin{itemize}
        \item Users can directly access and modify the structure of the list, \textbf{without letting the list know!}
        \item Even though methods of `create' and `destroy' are provided, memory management is still a problem because users may forget to call them (or fail to call them correctly).
        \item The name of every function starts with `\ttt{linked\_list}', which is lengthy and inconvenient.
    \end{itemize}
\end{frame}

\begin{frame}
    \begin{figure}[h]
        \centering
        \includegraphics[height=0.9\textheight]{figures/lamp.jpg}
    \end{figure}
\end{frame}

\begin{frame}[fragile]{Separate Implementation Details and Interfaces}
    \begin{columns}
        \begin{column}{0.5\linewidth}
            \begin{cpp}
struct Point2d {
 private:
  // implementation details
  double x, y;
 public:
  // interfaces
  void set_x(double new_x)
    { x = new_x; }
  void set_y(double new_y)
    { y = new_y; }
  double get_x()
    { return x; }
  double get_y()
    { return y; }
};
            \end{cpp}   
        \end{column}
        \begin{column}{0.5\linewidth}
            Access modifiers:
            \begin{itemize}
                \item \bluett{private}: Only the code inside the class \gray{(or in a \ttt{friend})} can access.
                \item \bluett{public}: Everyone can access.
                \item \gray{\ttt{protected}: Only the code inside the class or in a subclass, or in a \ttt{friend} can access.}
            \end{itemize}
        \end{column}
    \end{columns}
\end{frame}

\begin{frame}[fragile]{Separate Implementation Details and Interfaces}
    \begin{itemize}
        \item Implementation details should be invisible to others.
        \item Interafces are defined for others to use.
    \end{itemize}
    \begin{center}
        \begin{cpp}
// In C++, we can directly use the name Point2d without the struct keyword.
// The weird typedefs are not needed, either.
Point2d p;
p.x = 4.2;  // Error!
p.set_x(4.2);
p.set_y(3.5);
std::cout << "(" << p.get_x() << ", "
          << p.get_y() << ")" << std::endl;
        \end{cpp}    
    \end{center}
\end{frame}

\section{Basic Knowledge}

\begin{frame}[fragile]{\class or \struct?}
    In C++, \textbf{the only differences} between \class and \struct are
    \begin{itemize}
        \item \blue{Default access level} for a \class is \bluett{private}, while for a \struct is \bluett{public}.
        \begin{columns}
            \begin{column}{0.5\linewidth}
                \begin{cpp}
class Point {
  double x, y; // private here
  // Other members.
};
                \end{cpp}
            \end{column}
            \begin{column}{0.5\linewidth}
                \begin{cpp}
struct Point {
  double x, y; // public here
  // Other members.
};
                \end{cpp}
            \end{column}
        \end{columns}
        \item \gray{Default inheritance protection level for a \ttt{class} is \ttt{private}, while for a \ttt{struct} is \ttt{public}.}
    \end{itemize}
\end{frame}

\begin{frame}[fragile]{\const Member Functions}
    \begin{cpp}
void print_point(const Point2d &p) {
  std::cout << "(" << p.get_x() << ", "
            << p.get_y() << ")" << std::endl;
}
    \end{cpp}
    \begin{itemize}
        \item The parameter should be declared as \const reference, since it is not modified.
        \item However, the code above won't compile.
        \item \blue{We need to specify what we can do on \ttt{const} objects.}
    \end{itemize}
\end{frame}

\begin{frame}[fragile]{\const Member functions}
    \begin{columns}
        \begin{column}{0.5\linewidth}
            \begin{cpp}
struct Point2d {
 private:
  double x, y;
 public:
  void set_x(double new_x)
    { x = new_x; }
  void set_y(double new_y)
    { y = new_y; }
  double get_x() const
    { return x; }
  double get_y() const
    { return y; }
};
            \end{cpp}        
        \end{column}
        \begin{column}{0.5\linewidth}
            \begin{itemize}
                \item On a non-\const object, both \const members and non-\const members can be called.
                \item On a \const object, only the \const members can be called.
                \item A \const member function should not modify the data members.
            \end{itemize}
            Use \const whenever possible!
        \end{column}
    \end{columns}
\end{frame}

\begin{frame}[fragile]{The \this Pointer}
    Inside a member function, when we refer to the name of a member, we are in fact referring to it through the \this pointer.
    \begin{cpp}
class Point2d {
  double x, y;
 public:
  void set_x(double new_x) {
    this->x = new_x;    // equivalent to x = new_x;
  }
  // Other members.
};
    \end{cpp}
    \begin{itemize}
        \item \this is a pointer that points to the object itself. For example, in `\ttt{class Point2d}', \this is of type \ttt{Point2d *}.
    \end{itemize}
\end{frame}

\begin{frame}[fragile]{Name Lookup in \class}
    An exception to the name lookup rule:
    \begin{itemize}
        \item Inside a \class, all the class members are visible, no matter they are before or after the usage.
    \end{itemize}
    \begin{cpp}
class Point2d {
 public:
  void set_x(double new_x)
    { x = new_x; }      // OK: The member 'x' is visible here.
  void set_y(double new_y)
    { y = new_y; }
  double get_x() const
    { return x; }
  double get_y() const
    { return y; }
 private:
  double x, y;
};
    \end{cpp}
\end{frame}

\begin{frame}[fragile]{Defining Member Functions outside the \class}
    A member function can be defined outside the \class definition, \textbf{but must be declared inside the class}.
    \begin{columns}
        \begin{column}{0.5\linewidth}
            \begin{cpp}
class Point2d {
 public:
  void set_x(double new_x) {
    x = new_x;
  }
  void set_y(double new_y) {
    y = new_y;
  }
  double get_x() const;
  double get_y() const;
 private:
  double x, y;
};
            \end{cpp}        
        \end{column}
        \begin{column}{0.5\linewidth}
            \begin{itemize}
                \item Use \blue{operator\ttt{::}} to refer to a name in the class scope.
            \end{itemize}
            \begin{cpp}
double Point2d::get_x() const {
  return x;
}
double Point2d::get_y() const {
  return y;
}
            \end{cpp}
            \begin{itemize}
                \item The \const keyword, if needed, must appear at both declaration and definition. It is a part of the function type.
            \end{itemize}
        \end{column}
    \end{columns}
\end{frame}

\begin{frame}[fragile]{Reference to the Object itself}
    \begin{columns}
        \begin{column}{0.5\linewidth}
            \begin{cpp}
class Point2d {
 public:
  Point2d &set_x(double new_x) {
    x = new_x;
    return *this;
  }
  Point2d &set_y(double new_y) {
    y = new_y;
    return *this;
  }
  // Other members.
};
            \end{cpp}        
        \end{column}
        \begin{column}{0.5\linewidth}
            \begin{itemize}
                \item \ttt{set\_x} and \ttt{set\_y} returns a reference to the object itself (which is an \blue{lvalue}) by \ttt{return *this;}
                \item Then we can do:
                \begin{cpp}
p.set_x(4.2).set_y(3.5);
                \end{cpp}
            \end{itemize}
        \end{column}
    \end{columns}
\end{frame}

\section{Construction}

\begin{frame}[fragile]{Constructors}
    \blue{Constructors} (`ctors' for short) define the ways of initializing an object.
    \begin{columns}
        \begin{column}{0.5\linewidth}
            \begin{cpp}
class Point2d {
 public:
  Point2d(double a, double b)
    : x(a), y(b) {}
  Point2d() : x(0), y(0) {}
  // Other members.
};
            \end{cpp}
        \end{column}
        \begin{column}{0.5\linewidth}
            \begin{cpp}
// Initializing a Point2d object with x = 3.5, y = 4.2
Point2d p1(3.5, 4.2);
// Default-initialization calls the default ctor with no arguments.
Point2d p2;
// p2 is initialized with x = 0, y = 0.
            \end{cpp}
        \end{column}
    \end{columns}
\end{frame}

\begin{frame}[fragile]{Constructors}
    \begin{itemize}
        \item The name of a ctor is name of the class.
        \item The return-type is omitted, because a construction expression always returns the constructed object. The following statement will output \ttt{4.2}.
        \begin{cpp}
std::cout << Point2d(4.2, 3.5).get_x() << std::endl;
        \end{cpp}
        \item There is a \textbf{constructor initializer list} after the parameters, starting with `\ttt{:}', containing explicit initialization of \blue{several} data members, separated with commas.
        \item The constructors are often \blue{overloaded}, i.e. we may define several different ways of initialization.
    \end{itemize}
\end{frame}

\begin{frame}{The Ctor Initializer List}
    The most important part of a ctor is the \textbf{initializer list}.
    \begin{itemize}
        \item The ctor-init-list is the part where data members are initialized.
        \item \textbf{All the data members} will be initialized before the function body begins, \textbf{in the order in which they appear in the class definition}.
        \item If a data member does not appear in the ctor-init-list, it is \textbf{default-initialized}.
        \begin{itemize}
            \item For built-in types, default-initialization makes the object hold an undefined value.
            \item For class types, default-initialization is calling the \textbf{default-ctor} of that class.
        \end{itemize}
    \end{itemize}
\end{frame}

\begin{frame}[fragile]{Ctor-init-list: Examples}
    \begin{itemize}
        \item The following ctor is \textbf{bad}: (It is default-initializing the members, and then assigning values to them.)
        \begin{cpp}
Point2d(double a, double b) {
  x = a;
  y = b;
}
        \end{cpp}
        \item The following ctor is \textbf{misleading}:
        \begin{cpp}
Point2d(double a, double b) : y(b), x(a) {}
        \end{cpp}
        \item The following ctor first default-initializes \ttt{x}, and then initializes \ttt{y} with given value.
        \begin{cpp}
Point2d(double b) : y(b) {}
        \end{cpp}
    \end{itemize}
\end{frame}

\begin{frame}[fragile]{Ctor-init-list: Examples}
    A typical bug I made (years ago):
    \begin{cpp}
class Snake_game {
  std::vector<Food> foods;
  size_t num_food;
  // Other members
 public:
  Snake_game(size_t n, /* other args */)
    : num_food(n), foods(num_food) {}
  // Other members
};
    \end{cpp}
\end{frame}

\begin{frame}[fragile]{Importance of Using Ctor-init-list}
    It is strongly suggested to use ctor-init-lists \textbf{routinely}:
    \begin{itemize}
        \item Initialize the member directly, instead of first default-initialize it and then assign it with a value.
        \item Some members cannot be default-initialized or cannot be assigned.
        \begin{cpp}
class Foo {
  const int x;
  int y;
 public:
  Foo(int a, int b) {
    // Error: x cannot be assigned after initialization.
    x = a;
    y = b;
  }
};
        \end{cpp}
    \end{itemize}
\end{frame}

\begin{frame}[fragile]{Default Initialization}
    \begin{itemize}
        \item The \textbf{default ctor} is the ctor with no arguments. It defines the behavior when the object is \textbf{default-initialized} or \textbf{value-initialized}.
        \item The following code outputs `\ttt{Liu Big God is so strong.}'.
        \begin{cpp}
class Point2d {
 public:
  Point2d() : x(0), y(0) {
    std::cout << "Liu Big God is so strong.\n"
  }
  // Other members.
};
// in main
Point2d p;
        \end{cpp}
    \end{itemize}
\end{frame}

\begin{frame}[fragile]{Default Initialization}
    \begin{cpp}
class Line2d {
  Point2d p0, v;
 public:
  Line2d()
    { std::cout << "Liu Big God is so powerful.\n" }
  // Other members.
};
    \end{cpp}
    What's the output of the following code (inside the \ttt{main} function)?
    \begin{cpp}
Line2d line;
    \end{cpp}
    \pause
    \bluett{Liu Big God is so strong.}\\
    \bluett{Liu Big God is so strong.}\\
    \bluett{Liu Big God is so powerful.}
\end{frame}

\begin{frame}[fragile]{Have a Try}
    Define a class \ttt{Vector}:
    \begin{itemize}
        \item Define the data members \ttt{m\_size}, \ttt{m\_capacity} and \ttt{m\_data}, denoting the number of elements stored, the maximum possible size of the current storage, and a pointer to the storage.
        \begin{itemize}
            \item The prefix \ttt{m\_} means `member'.
        \end{itemize}
        \item Define a default-ctor that initializes the Vector to an empty Vector, i.e. \ttt{m\_size = 0}, \ttt{m\_capacity = 0}, \ttt{m\_data = \blue{nullptr}}.
        \item Define a ctor that receives an unsigned number \ttt{n}, which initializes the Vector to be holding \ttt{n} value-initialized \bluett{int}s.
        \item Define a ctor that receives two pointers \ttt{begin} and \ttt{end} pointing to the beginning and the pass-end of an array. The ctor copies the values from the range of the array.
    \end{itemize}
\end{frame}

\begin{frame}[fragile]{Constructors}
    \begin{cpp}
class Vector {
  size_t m_size, m_capacity;
  int *m_data;
 public:
  Vector() : m_size(0), m_capacity(0), m_data(nullptr) {}
  Vector(size_t n) : m_size(n), m_capacity(n),
    m_data(new int[n]()) {}
  Vector(int *begin, int *end)
    : m_size(end - begin), m_capacity(end - begin),
      m_data(new int[end - begin]()) {
    for (int *p = m_data; begin != end; ++begin, ++p)
      *p = *begin;
  }
};
    \end{cpp}
\end{frame}

\begin{frame}{\bluett{nullptr}}
    \begin{itemize}
        \item In C, the null pointer \ttt{NULL} is defined to be of type \bluett{void }\ttt{*}, so that it can be converted to any pointer type.
        \item However, C++ does not allow implicit conversion from \bluett{void }\ttt{*} to other pointer types. It's possible that \ttt{NULL} is defined as \ttt{(\blue{long})0}.
        \begin{itemize}
            \item C++ allows initialization of a pointer with integral literal \ttt{0}, but not other values.
        \end{itemize}
        \item Since C++11, we use \bluett{nullptr} as a well-defined null pointer. It is of type \ttt{std::}\bluett{nullptr\_t}, a type defined in \ttt{<cstddef>}. (\ttt{stddef.h} does not have this!)
    \end{itemize}
\end{frame}

\newcommand{\new}{\bluett{new}~}
\newcommand{\delete}{\bluett{delete}~}

\begin{frame}{\new and \delete}
    \begin{itemize}
        \item Roughly speaking, \new and \delete are like the C++ version of \ttt{malloc} and \ttt{free}: They allocate memory on heap.
        \item \new not only \textbf{allocates the memory}, but also \textbf{constructs the object}.
        \item \delete first \textbf{destructs the object}, and then \textbf{deallocates (frees) the memory}.
        \item \new and \delete will call the ctor and dtor of the type, while \ttt{malloc} and \ttt{free} won't!
        \item When you have some object on heap in C++, \textbf{NEVER use \ttt{malloc}/\ttt{calloc}/\ttt{realloc}/\ttt{free}}. You should always use \new and \delete.
    \end{itemize}
\end{frame}

\begin{frame}[fragile]{\new and \delete}
    \begin{cpp}
int *pi = new int;                // default initialization
int *pi2 = new int();             // value initialization
int *pia = new int[10];           // default initialization
int *pia2 = new int[10]();        // value initialization
int *pia3 = new int[10]{1, 2, 3}; // First three values: 1, 2, 3
                                  // Others: value initialization
// call Point2d::Point2d(double, double)
Point2d *p = new Point2d(3, 4);
// call Point2d::Point2d()
Point2d *p2 = new Point2d;
// call Point2d::Point2d()
Point2d *p3 = new Point2d();
    \end{cpp}
\end{frame}

\begin{frame}{Initialization in \new}
    \begin{itemize}
        \item Whenever the explicit initializer is absent, \blue{default initialization} is performed.
        \item Whenever an explicit initializer is provided but as an empty pair of parentheses, \blue{value initialization} is performed.
        \item For a class type \textbf{with a user-defined default ctor}, both default initialization and value initialization call the default ctor.
    \end{itemize}
    The rules for initialization are \textbf{very complicated}. Now you only need to know these that are listed above.
\end{frame}

\begin{frame}[fragile]{\new and \delete}
    \begin{itemize}
        \item To destroy the object created by \new and free the memory, use \delete.
        \begin{cpp}
int *pi = new int();
// after some operations
delete pi;
        \end{cpp}
        \item For an array type, use \bluett{delete}\ttt{[]}.
        \begin{cpp}
int *pia = new int[10]();
// after some operations
delete[] pia;
        \end{cpp}
        \item \delete will call the destructor of a class type. (We will talk about destructors later.)
    \end{itemize}
\end{frame}

\begin{frame}[fragile]{In-class Initializer}
    \begin{columns}
        \begin{column}{0.5\linewidth}
            \begin{itemize}
                \item An \blue{in-class initializer} is used to define the default value for a member.
                \item When a member with an in-class initializer is \blue{default initialized} or \blue{value initialized}, it will be initialized with the in-class initializer.
            \end{itemize}
        \end{column}
        \begin{column}{0.5\linewidth}
            \begin{cpp}
class Point2d {
  double x = 0, y = 0;
  // Other members.
};

class Line2d {
  // Curly braces are allowed here since C++11.
  Point2d p0{0, 0}, v{1, 1};
  // Other members.
};
            \end{cpp}
        \end{column}
    \end{columns}
\end{frame}

\begin{frame}[fragile]{In-class Initializer}
    Due to the limitation of compilers (you will learn about it in CS131), parentheses are not allowed in in-class initializer: They will be viewed as function declarations!
    \begin{cpp}
class Line2d {
  // Error: They are seen as function declarations.
  Point2d p0(0, 0), v(1, 1);
  int foo(); // Exactly a member function declaration.
  // Other members.
};
    \end{cpp}
\end{frame}

\begin{frame}{Synthesized Default Ctor}
    The compiler will generate a default ctor if the following conditions are satisfied:
    \begin{itemize}
        \item Each member without a in-class initializer is \blue{default-initializable}, and
        \item There is no user-defined ctors, or the default ctor is defined as \ttt{=default}.
    \end{itemize}
    Many kinds of members are not \blue{default-initializable}, e.g. a reference, or a class type member whose class does not have accessible default ctor.
\end{frame}

\begin{frame}{Synthesized Default Ctor}
    The synthesized default ctor will
    \begin{itemize}
        \item initialize a member with the in-class initializer, if it has an in-class initializer.
        \item Otherwise, it will default-initialize the member.
        \item It is \bluett{public}.
    \end{itemize}
\end{frame}

\begin{frame}[fragile]{Delegating Ctor}
    Since C++11, a ctor can \blue{delegate} part or all of its work to another ctor:
    \begin{cpp}
class Point2d {
  double x, y;
 public:
  Point2d() : Point2d(0, 0) {
    std::cout << "default ctor body\n";
  }
  Point2d(double a, double b) : x(a), y(b) {
    std::cout << "Point2d(double, double) body\n";
  }
};
// in main
Point2d p;
    \end{cpp}
    The above code outputs ``\bluett{Point2d(double, double) body}'' first, and then ``\bluett{default ctor body}''.
\end{frame}

\begin{frame}[fragile]{Example}
    \begin{cpp}
class Vector {
  size_t m_size = 0, m_capacity = 0;
  int *m_data = nullptr;
 public:
  // explicitly require a synthesized default ctor
  Vector() = default;
  Vector(size_t n) : m_size(n), m_capacity(n),
    m_data(new int[n]()) {}
  // delegate to Vector::Vector(size_t)
  Vector(int *begin, int *end) : Vector(end - begin) {
    for (int *p = m_data; begin != end; ++begin, ++p)
      *p = *begin;
  }
};
    \end{cpp}
\end{frame}

\section{Destruction}

\begin{frame}{Destructor}
    \begin{definition}[Destructor]
        Destructor (`dtor' for short) is a member function which is called automatically when the object is destroyed.
    \end{definition}
    Several typical cases:
    \begin{itemize}
        \item When control flow reaches the end of the scope of a local object.
        \item When the object (if it is \bluett{new}ed) is \bluett{delete}d.
        \item \gray{When an exception happens...}
    \end{itemize}
\end{frame}

\begin{frame}{Define a Dtor}
    \begin{itemize}
        \item A dtor does not return a value. The value-type is omitted.
        \item The name of a dtor is \ttt{\~}\textit{class-name}. E.g. \ttt{\~{}Point2d}.
        \item The dtor is unique and cannot be overloaded. A dtor receives no arguments.
        \item The function body of the dtor does the work that needs to be done when an object is destroyed, e.g. \blue{frees the resources} or output something.
        \item \textbf{After the function body is executed}, the members are destroyed by calling their own dtors, \textbf{in reverse order} from the order in which they are declared.
    \end{itemize}
\end{frame}

\begin{frame}[fragile]{Define a Dtor}
    \begin{columns}
        \begin{column}{0.5\linewidth}
            The dtor of the \ttt{Point2d} class doesn't need to do anything.
            \begin{cpp}
class Point2d {
 public:
  ~Point2d() {}
  // Other members.
};
            \end{cpp}
        \end{column}
        \begin{column}{0.5\linewidth}
            The dtor of \ttt{Vector} should free the memory it holds.
            \begin{cpp}
class Vector {
 public:
  ~Vector() { delete[] m_data; }
  // Other members.
};
            \end{cpp}
        \end{column}
    \end{columns}
\end{frame}

\begin{frame}{Access Level of Ctors and Dtors}
    \begin{itemize}
        \item The access restrictions also apply to ctors and dtors. For example, a \bluett{private} ctor can only be accessed inside the class or in its \bluett{friend}s.
        \item At any part of the code, if the \textbf{dtor} of a class is invisible, then \textbf{constructing} an object of such type is not allowed. (If an object is not destructible, then it is not constructible.)
    \end{itemize}
\end{frame}

\begin{frame}{Synthesized Dtors}
    \begin{itemize}
        \item If there is no user-defined dtor, the compiler will generate one by default.
        \item The synthesized dtor is
        \begin{itemize}
            \item \bluett{public},
            \item with an empty function body,
            \item \gray{non-virtual},
            \item \gray{and implicitly \ttt{noexcept}}.
        \end{itemize}
    \end{itemize}
\end{frame}

\begin{frame}[fragile]{Destructing Order}
    \begin{columns}
        \begin{column}{0.5\linewidth}
            \begin{cpp}
class Point2d {
 public:
  ~Point2d() {
    std::cout<< "Point dtor.\n";
  }
  // Other members.
};
class Line2d {
  Point2d p0{0, 0}, v; 
 public:
  ~Line2d() {
    std::cout << "Line dtor.\n";
  }
  // Other members.
};
            \end{cpp}
        \end{column}
        \begin{column}{0.5\linewidth}
            What's the output of the following code?
            \begin{cpp}
int main() {
  Point2d *p = new Point2d();
  delete p;
  Line2d l;
  return 0;
}
            \end{cpp}
            \pause
            \bluett{Point dtor.}\\
            \bluett{Line dtor.}\\
            \bluett{Point dtor.}\\
            \bluett{Point dtor.}
        \end{column}
    \end{columns}
\end{frame}

\section{Further Reading}

\begin{frame}{Further Reading}
    \begin{itemize}
        \item Effective C++ Item 18: Make interfaces easy to use correctly and hard to use incorrectly.
        \begin{itemize}
            \item Read until the \ttt{class Month} example (about three pages). The rest part is not for you now.
        \end{itemize}
        \item Effective C++ Item 22: Declare data members private.
    \end{itemize}
\end{frame}

\end{document}