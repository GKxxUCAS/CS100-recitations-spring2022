\documentclass{beamer}

\usetheme{Berlin}
\usecolortheme{lily}

\usepackage{../cppenv}
\usepackage{../recdefs}

\usepackage{url}

\title{Overloaded and Customized \ttt{new}/\ttt{delete}}
\author{GKxx}
\date{\today}

\AtBeginSubsection{
  \begin{frame}{Contents}
    \tableofcontents[currentsection, currentsubsection]
  \end{frame}
}

\begin{document}

\begin{frame}
	\maketitle
\end{frame}

\begin{frame}
  \tableofcontents
\end{frame}

\section{\ttt{new} Expressions and \ttt{operator new}}

\begin{frame}{\ttt{new} Expressions}
  The execution of a \bluett{new} expression takes two steps:
  \begin{enumerate}
    \item Allocate a block of memory.
    \item Construct the object(s) on the allocated memory.
  \end{enumerate}
  What we can control is the first step.
\end{frame}

\begin{frame}[fragile]{\ttt{operator new}}
  Memory allocation is done by a group of functions:
  \begin{cpp}
// Not inlined, not in any namespace
void *operator new(std::size_t size);
void *operator new[](std::size_t size);
  \end{cpp}
  \begin{itemize}
    \item For \ilcpp{new Type(args)}, the memory is allocated by calling \ilcpp{operator new(sizeof(Type))}.
    \item For \ilcpp{new Type[n]\{initializers\}}, the memory is allocated by calling \ilcpp{operator new[](sizeof(Type) * n)}.
    \item[*]{\footnotesize C++17 \blue{alignment-aware allocation}? Talk later.}
  \end{itemize}
\end{frame}

\begin{frame}[fragile]{\ttt{operator new}}
  \begin{cpp}
void *operator new(std::size_t size);
void *operator new[](std::size_t size);
  \end{cpp}
  \begin{itemize}
    \item These two functions \textbf{do not know} the type of object(s) to be created.
    \item \ilcpp{operator new[]} \textbf{does not know} the number of objects to be created.
  \end{itemize}
\end{frame}

\begin{frame}{\ttt{delete} Expressions}
  The execution of a \bluett{delete} expression takes two steps:
  \begin{enumerate}
      \item Destroy the object. {\small(Not executed by C++20 \blue{destroying-delete})}
    \item Deallocate the memory.
  \end{enumerate}
  What we can control is the second step.
\end{frame}

\begin{frame}[fragile]{\ttt{operator delete}}
  Memory deallocation is done by a group of functions:
  \begin{cpp}
// Not inlined, not in any namespace
void operator delete(void *ptr) noexcept;
void operator delete[](void *ptr) noexcept;
  \end{cpp}
  \begin{itemize}
    \item \ilcpp{delete ptr} deallocates the memory by calling \ilcpp{operator delete(ptr)}.
    \item \ilcpp{delete[] ptr} deallocates the memory by calling \ilcpp{operator delete[](ptr)}.
    \item[*]{\footnotesize C++14 \blue{sized-deallocation}? Talk later.}
  \end{itemize}
\end{frame}

\begin{frame}[fragile]{\ttt{operator delete}}
  Exceptions are not welcomed!
  \begin{itemize}
    \item All deallocation functions are \bluett{noexcept}, unless specified otherwise in the declaration.
    \item If a deallocation function terminates by throwing an exception, the behavior is \textbf{undefined}, even if it is declared with \bluett{noexcept(false)}.
    \begin{itemize}
      \item Such exception is not expected to be caught. Stack is possibly not unwound in this case.
    \end{itemize}
  \end{itemize}
\end{frame}

\section{Overloading \ttt{operator new}}

\subsection{Standard Library Version}

\begin{frame}[fragile]{Standard \ttt{operator new}}
  The following functions are \textit{replacable}:
  \begin{cpp}
void *operator new(std::size_t size);
void *operator new[](std::size_t size);
void operator delete(void *ptr) noexcept;
void operator delete[](void *ptr) noexcept;
  \end{cpp}
  \begin{itemize}
    \item Standard versions (normal versions) are defined in standard library file \ttt{<new>}.
    \item But the compiler will choose the user-defined version if there exists one.
    \item In this case, they do not constitute redefinition.
  \end{itemize}
\end{frame}

\begin{frame}{Standard \ttt{operator new}}
  Difference between \bluett{operator new} and \ttt{malloc}?
  \pause
  \par Basic:
  \begin{itemize}
    \item \bluett{operator new} allocates some memory when \ilcpp{size == 1}, while the behavior of \ilcpp{malloc(0)} is implementation-defined.
    \item On failure, \bluett{operator new} throws \ttt{std::bad\_alloc}, while \ttt{malloc} returns null pointer.
  \end{itemize}
\end{frame}

\begin{frame}[fragile]{Standard \ttt{operator new}}
  A simple \bluett{operator new} that uses \ttt{malloc} for allocation:
  \begin{cpp}
void *operator new(std::size_t size) {
  if (size == 0)
    size = 1;
  if (auto ptr = std::malloc(size))
    return ptr;
  throw std::bad_alloc{};
}
  \end{cpp}
  (Similar for \bluett{operator new}\ttt{[]}\dots)
\end{frame}

\begin{frame}[fragile]{Standard \ttt{operator new}}
  In fact, \bluett{operator new} keeps trying to allocate memory and, on failure, does some possible adjustment by calling a \textbf{new-handler}, until the allocation succeeds or no new-handler is available.
  \begin{cpp}[\scriptsize]
void *operator new(std::size_t size) {
  if (size == 0)
    size = 1;
  while (true) {
    if (auto ptr = std::malloc(size))
      return ptr;
    auto handler = std::get_new_handler();
    if (handler)
      handler();
    else
      throw std::bad_alloc{};
  }
}
  \end{cpp}
\end{frame}

\begin{frame}[fragile]{Standard \ttt{operator delete}}
  Possible implementation of \bluett{operator delete} that uses \ttt{std::free} to deallocate memory:
  \begin{cpp}
void operator delete(void *ptr) noexcept {
  std::free(ptr);
}
  \end{cpp}
  \begin{itemize}
    \item Make sure it is safe to \bluett{delete} a null pointer.
    \item Similar for \bluett{operator delete}\ttt{[]}.
  \end{itemize}
\end{frame}

\subsection{Overloading}

\begin{frame}{Why Overload them?}
  \textit{Effective C++} Item 50 talks about the following most common reasons:
  \begin{itemize}
    \item To detect usage errors.
    \item To improve efficiency.
    \item To collect usage statistics.
  \end{itemize}
\end{frame}

\begin{frame}[fragile]{Record Allocations}
  \begin{cpp}
void *operator new(std::size_t size) {
  if (size == 0)
    size = 1;
  while (true) {
    if (auto ptr = std::malloc(size)) {
      recorder.add_record(ptr);
      return ptr;
    }
    auto handler = std::get_new_handler();
    if (handler)
      handler();
    else
      throw std::bad_alloc{};
  }
}
  \end{cpp}
\end{frame}

\begin{frame}[fragile]{Record Allocations}
  \begin{cpp}
void operator delete(void *ptr) noexcept {
  if (!recorder.find(ptr))
    throw std::invalid_argument
        {"Invalid pointer passed to operator delete"};
  recorder.remove_record(ptr);
  std::free(ptr);
}
  \end{cpp}
\end{frame}

\begin{frame}[fragile]{Class-specific Versions}
  \begin{cpp}[\small]
struct Widget {
  static void *operator new(std::size_t size);
  static void *operator new[](std::size_t size);
  static void operator delete(void *ptr);
  static void operator delete[](void *ptr);
};
  \end{cpp}
  \begin{itemize}
    \item When we use \bluett{new}/\bluett{new[]} to create class-type objects, the lookup for \bluett{operator new}/\bluett{operator new}\ttt{[]} begins in the class scope.
    \item If the \bluett{new}-expression uses the form \ilcpp{::new}, the class-scope lookup is \textbf{bypassed} and the global version \ilcpp{::operator new}/\ilcpp{::operator new[]} will be called.
  \end{itemize}
\end{frame}

\begin{frame}[fragile]{Class-specific Versions}
  \begin{cpp}[\small]
struct Widget {
  static void *operator new(std::size_t size);
  static void *operator new[](std::size_t size);
  static void operator delete(void *ptr);
  static void operator delete[](void *ptr);
};
  \end{cpp}
  \begin{itemize}
    \item The keyword \bluett{static} is optional: these functions are always static members.
    \item Deallocation functions are implicitly \bluett{noexcept}.
  \end{itemize}
\end{frame}

\begin{frame}[fragile]{Example: \ttt{Heap\_tracked}}
  This example is from \textit{More Effective C++} Item 27: Requiring or prohibiting heap-based objects.
  \begin{itemize}
    \item \ilcpp{dynamic_cast<const void *>(ptr)} yields the beginning address of the object. (Casting it to \ilcpp{void *}, \ilcpp{volatile void *} or \ilcpp{const volatile void *} also work.)
    \item Track whether an object is heap-based by inheriting \ttt{Heap\_tracked} in a \textbf{mixin} style.
  \end{itemize}
\end{frame}

\section{Placement-\ttt{new}}

\subsection{Standard Library Versions}

\begin{frame}[fragile]{\ttt{new}/\ttt{delete} with Extra Arguments}
  \begin{cpp}[\scriptsize]
void *operator new(std::size_t size, const std::nothrow_t &) noexcept;
void *operator new[](std::size_t size, const std::nothrow_t &) noexcept;
void *operator new(std::size_t size, void *place) noexcept;
void *operator new[](std::size_t size, void *place) noexcept;

void operator delete(std::size_t size, const std::nothrow_t &) noexcept;
void operator delete[](std::size_t size, const std::nothrow_t &) noexcept;
void operator delete(std::size_t size, void *place) noexcept;
void operator delete[](std::size_t size, void *place) noexcept;
  \end{cpp}
\end{frame}

\begin{frame}[fragile]{Non-throwing \ttt{operator new}}
  \begin{cpp}
auto ptr = new (std::nothrow) Type(args);
auto arr = new (std::nothrow) Type[n];
  \end{cpp}
  \begin{itemize}
    \item \ilcpp{std::nothrow} is a tag of type \ilcpp{std::nothrow\_t} defined in \ttt{<new>}.
    \begin{cpp}
namespace std {
  struct nothrow_t {
    explicit nothrow_t() = default;
  };
  extern const nothrow_t nothrow;
}
    \end{cpp}
  \end{itemize}
\end{frame}

\begin{frame}[fragile]{Non-throwing \ttt{operator new}}
  \begin{cpp}
auto ptr = new (std::nothrow) Type(args);
auto arr = new (std::nothrow) Type[n];
  \end{cpp}
  \begin{itemize}
    \item \ilcpp{new (std::nothrow) Type(args)} calls \ilcpp{operator new(sizeof(Type), std::nothrow)} for memory allocation.
    \item \ilcpp{new (std::nothrow) Type[n]\{initializers\}} calls \ilcpp{operator new[](sizeof(Type) * n, std::nothrow)} for memory allocation.
    \item Returns null pointer on failure. No exception would be thrown.
  \end{itemize}
\end{frame}

\begin{frame}[fragile]{Non-throwing \ttt{operator new}}
  Possible implementation:
  \begin{cpp}
void *operator new(std::size_t size,
                   const nothrow_t &) noexcept {
  void *ptr = nullptr;
  try {
    ptr = ::operator new(size);
  } catch (...) {}
  return ptr;
}
  \end{cpp}
\end{frame}

\begin{frame}[fragile]{Placement-\ttt{new}}
  The ``real'' placement-\bluett{new}:
  \begin{cpp}
Type *pos1 = somewhere();
new (pos1) Type(args);
Type *pos2 = somewhere_else();
new (pos2) Type[n]{@\textit{a,b,c,...}@};
  \end{cpp}
  \begin{itemize}
    \item No allocation is performed.
    \item Placement-\bluett{new} is used for construct object(s) on given place.
  \end{itemize}
\end{frame}

\begin{frame}[fragile]{Placement-\ttt{new}}
  Possible implementation:
  \begin{cpp}
void *operator new(std::size_t, void *place) noexcept {
  return place;
}
void *operator new[](std::size_t, void *place) noexcept {
  return place;
}
  \end{cpp}
  \begin{notice}
    These two functions (as well as the corresponding \bluett{operator delete}s) \textbf{cannot} be replaced.
  \end{notice}
\end{frame}

\subsection{Customized Versions}

\begin{frame}[fragile]{Customized Arguments}
  \begin{cpp}
void *operator new(std::size_t size,
                   long line, const char *file) {
  auto ptr = ::operator new(size);
  log_allocation(ptr, size, line, file);
  return ptr;
}
auto ptr = new (__LINE__, __FILE__) Type(args);
  \end{cpp}
\end{frame}

\begin{frame}[fragile]{Placement-\ttt{delete}}
  Recall the two steps for a \bluett{new} expression:
  \begin{enumerate}
    \item Allocate enough memory.
    \item Construct the object(s).
  \end{enumerate}
  For a \bluett{new} expression \ilcpp{new (args...) Type(ctor_args...)}, if an exception is thrown during the \textbf{second} step:
  \begin{itemize}
    \item The corresponding \bluett{operator delete} is called with \ilcpp{ptr, args...} passed to it, where \ilcpp{ptr} is the beginning location of memory allocated in the first step.
    \item The \bluett{operator delete} deallocates the memory allocated by \bluett{operator new} to ensure memory-safety and exception-safety.
  \end{itemize}
\end{frame}

\begin{frame}[fragile]{Placement-\ttt{delete}}
  Possible implementation for non-throwing \bluett{new}:
  \begin{cpp}
void operator delete(void *ptr,
                     const std::nothrow_t &) noexcept {
  ::operator delete(ptr);
}
  \end{cpp}
  \pause
  Possible implementation of placement-\bluett{delete} for our customized placement-\bluett{new}:
  \begin{cpp}
void operator delete(void *ptr,
            long line, const char *file) noexcept {
  log_failure(ptr, line, file);
  ::operator delete(ptr);
}
  \end{cpp}
\end{frame}

\begin{frame}[fragile]{Placement-\ttt{delete}}
  Possible implementation for the real ``placement-\bluett{new}''?
  \pause
  \begin{cpp}
void operator delete(void *, void *) noexcept {}
void operator delete[](void *, void *) noexcept {}
  \end{cpp}
  \pause
  \begin{notice}
    If no suitable placement-\bluett{delete} is found, no deallocation function would be called, which possibly results in memory leak.
  \end{notice}
\end{frame}

\begin{frame}[fragile]{Placement-\ttt{delete}}
  Which \ilcpp{operator delete} is called?
  \begin{cpp}
auto ptr = new (std::nothrow) Type(args);
delete ptr;
  \end{cpp}
  \pause
  Answer: \textbf{the normal version with no extra arguments.} A placement-\bluett{delete} is called only when constructors throw an exception.
\end{frame}

\section{\ttt{new}/\ttt{delete} in Modern C++}

\subsection{Sized-\ttt{delete}}

\begin{frame}{Sized-\ttt{delete}}
  For some kinds of deallocation functions, it might be necessary to know the \textbf{size} of the block of memory to be deallocated.\par
  A deallocation function which receives an extra \ttt{std::size\_t} parameter is a \textbf{sized-deallocation function}.
  \begin{notice}
    Sized-deallocation function is a \textbf{usual} deallocation function. It is not a placement version.
  \end{notice}
\end{frame}

\begin{frame}[fragile]{Sized-\ttt{delete}}
  Before C++14, sized-\bluett{delete} could only be class-scoped static member:
  \begin{cpp}
struct Widget {
  int x;
  static void operator delete(void *p, std::size_t sz) {
    std::cout << "size == " << sz << '\n';
    ::operator delete(p);
  }
};
auto ptr = new Widget;
delete ptr;
  \end{cpp}
  Output:
  \begin{txt}
size == 4
  \end{txt}
\end{frame}

\begin{frame}[fragile]{Sized-\ttt{delete}}
  Before C++14, sized-\bluett{delete} could only be class-scoped static member:
  \begin{cpp}
struct Widget {
  int x;
  static void operator delete[](void *p, std::size_t sz) {
    std::cout << "size == " << sz << '\n';
    ::operator delete[](p);
  }
};
auto ptr = new Widget[100];
delete []ptr;
  \end{cpp}
  Possible output:
  \begin{txt}
size == 408
  \end{txt}
\end{frame}

\begin{frame}[fragile]{Sized-\ttt{delete}}
  If a sized-deallocation function is defined and its corresponding unsized version is not, the sized version is called for a \bluett{delete}-expression to deallocate the memory.
  \begin{itemize}
    \item The \ttt{std::size\_t} argument is passed by the compiler automatically.
  \end{itemize}
  \pause
  Since C++14, global sized-deallocation functions are also allowed:
  \begin{cpp}[\scriptsize]
void operator delete(void *ptr, std::size_t size) noexcept;
void operator delete[](void *ptr, std::size_t size) noexcept;
  \end{cpp}
\end{frame}

\begin{frame}{Sized-\ttt{delete}}
  The compiler may choose to call the sized version \textbf{or} the unsized one.
  \begin{itemize}
    \item Clang-14 calls the unsized version by default, even when the sized version is provided.
    \item Calls to one version must be effectively equivalent to the other version, otherwise the program has undefined behavior.
    \item The standard library implementations of sized-deallocation functions directly call the unsized versions.
  \end{itemize}
\end{frame}

\subsection{Alignment-aware \ttt{new}/\ttt{delete}}

\begin{frame}{Alignment of Objects}
  Every object type has the \textbf{alignment requirement} property, representing the number of bytes between successive addresses at which objects of this type can be allocated.
  \begin{itemize}
    \item Alignment requirement of an object is an integer value of type \ttt{std::size\_t}, and is always a power of 2.
    \item Alignment requirement could be queried with \bluett{alignof} or \ttt{std::alignment\_of}.
  \end{itemize}
\end{frame}

\begin{frame}[fragile]{Alignment of Objects}
  On 64-bit Ubuntu 20.04:
  \begin{itemize}
    \item \ilcpp{alignof(int)}: 4
    \item \ilcpp{alignof(long)}: 8
    \item \ilcpp{alignof(char)}: 1
  \end{itemize}
  \pause
  \begin{cpp}
struct Widget {
  int x;
  char y;
};
  \end{cpp}
  \ilcpp{alignof(Widget)} is 4 because \ttt{x} must be placed at 4-byte boundaries.
\end{frame}

\begin{frame}[fragile]{Alignment of Objects}
  We may use \ilcpp{alignas} to set a special alignment requirement:
  \begin{cpp}
struct alignas(32) Widget {
  // ...
};
  \end{cpp}
  \pause
  Some types may have special alignment requirements: Intel intrinsic type \ttt{\_\_m256} is a 256-bit type and is aligned at 32-byte boundaries.
\end{frame}

\begin{frame}[fragile]{Alignment-aware Allocation}
  Since C++17, a group of alignment-aware allocation functions are introduced:
  \begin{cpp}[\scriptsize]
void *operator new(std::size_t size, std::align_val_t al);
void *operator new[](std::size_t size, std::align_val_t al);
void *operator new(std::size_t size,
      std::align_val_t al, const std::nothrow_t &) noexcept;
void *operator new[](std::size_t size,
      std::align_val_t al, const std::nothrow_t &) noexcept;
  \end{cpp}
  (Together with their deallocation functions and sized-deallocation functions.)
  \begin{itemize}
    \item \ttt{<cstdlib>} also introduces \ttt{std::aligned\_alloc}.
  \end{itemize}
\end{frame}

\begin{frame}[fragile]{Alignment-aware Allocation}
  \begin{cpp}
namespace std {
  enum class align_val_t : size_t {};
}
  \end{cpp}
  \begin{itemize}
    \item Normal allocation functions allocate objects aligned at \ttt{\_\_STDCPP\_DEFAULT\_NEW\_ALIGNMENT\_\_} (might be 16).
    \item If \ilcpp{alignof(Type)} exceeds the default \bluett{new} alignment, the \bluett{new}-expression calls the alignment-aware \bluett{operator new} and passes \ilcpp{alignof(Type)} as the second argument. (Similar for array-\bluett{new}.)
  \end{itemize}
\end{frame}

\subsection{Destroying-\ttt{delete}}

\begin{frame}[fragile]{Destroying-\ttt{delete}}
  \begin{cpp}
namespace std {
  struct destroying_delete_t {
    explicit destroying_delete_t() = default;
  };
  inline constexpr destroying_delete_t
      destroying_delete{}; // a tag
}
struct T {
  void operator delete(T *ptr, std::destroying_delete_t);
  // Together with its sized and alignment-aware versions.
};
  \end{cpp}
\end{frame}

\begin{frame}[fragile]{Destroying-\ttt{delete}}
  If a destroying-\bluett{delete} is defined:
  \begin{itemize}
    \item \bluett{delete}-expressions do not execute the destructor before a call to \bluett{operator delete}.
    \item The destroying-\bluett{delete} is chosen in preference to the normal \bluett{operator delete}.
    \item It becomes the responsibility of the destroying-\bluett{delete} to destroy the object correctly.
  \end{itemize}
  \pause
  What for? See \url{https://open-std.org/JTC1/SC22/WG21/docs/papers/2017/p0722r1.html}.
\end{frame}

\end{document}