\documentclass{beamer}

\usepackage{../cppenv}
\usepackage{../recdefs}

\usetheme{Singapore}

\title{CS100 Recitation 9}
\author{GKxx}
\date{April 18, 2022}

\AtBeginSubsection{
    \begin{frame}{Contents}
        \tableofcontents[currentsection, currentsubsection]
    \end{frame}
}

\begin{document}

\begin{frame}
    \maketitle
\end{frame}

\section{Inheritance and Polymorphism}

\subsection{Inheritance}

\begin{frame}{Defining a Subclass}
    An item for sale:
    \begin{itemize}
        \item \ttt{std::string name;}
        \item \ttt{double price;}
        \item \ttt{std::string get\_name() const;}
        \item \ttt{double net\_price(std::size\_t n) const;}
    \end{itemize}
    A discounted item \textbf{is an} item, and has some more information:
    \begin{itemize}
        \item \ttt{std::size\_t min\_quantity;}
        \item \ttt{double discount;}
    \end{itemize}
    The net price for such item is \ttt{n * price} if \ttt{n < min\_quantity}, or \ttt{n * discount * price} otherwise.
\end{frame}

\begin{frame}{Defining a Subclass}
    Things to consider:
    \begin{itemize}
        \item Does your class need a default constructor?
        \begin{itemize}
            \item If so, what should be a reasonable behavior?
            \item What will happen if not?
        \end{itemize}
        \item Does your class need special copy-control?
        \begin{itemize}
            \item Seems not.
            \item But what if we have another thing called a \ttt{Basket}...?
            \item What if every item has a unique id...?
        \end{itemize}
        \item What value should \ttt{discount} have to represent `20\% off'?
    \end{itemize}
\end{frame}

\begin{frame}{\bluett{protected} members}
    A \bluett{protected} member is private, except that it is accessible in subclasses.
    \begin{itemize}
        \item \ttt{price} is accessible in \ttt{Discounted\_item}.
        \item Should \ttt{name} be \bluett{protected} or \bluett{private}?
        \begin{itemize}
            \item \private is ok if the subclass doesn't (shouldn't) modify it. It is accessible through the public \ttt{get\_name} interface.
            \item \bluett{protected} is also reasonable.
        \end{itemize}
    \end{itemize}
    The core idea is to \textbf{separate implementation details and interfaces}.
\end{frame}

\begin{frame}[fragile]{Inheritance}
    By defining \ttt{Discounted\_item} to be a subclass of \ttt{Item}, \textbf{every object of \ttt{Discounted\_item} contains an object of \ttt{Item}}.
    \begin{itemize}
        \item Every data member and member function, except the constructors, are inherited, no matter what access level they have.
        \item What can we derive from this?
        \begin{itemize}
            \item When constructing an object of a subclass, one of the ctors of the base class must be called before initializing the members that the subclass declares.
            \item The dtor of the subclass must call the dtor of the base class (automatically) after the members of the subclass are destroyed.
            \item \ttt{\blue{sizeof}(Derived) >= \blue{sizeof}(Base)}.
        \end{itemize}
    \end{itemize}
\end{frame}

\begin{frame}{Inheritance}
    Core ideas of inheritance:
    \begin{itemize}
        \item Every sub-object contains an object of the base class.
        \item The father has his own ways of doing things, which children cannot affect!
    \end{itemize}
\end{frame}

\begin{frame}[fragile]{Inheritance and Constructors}
    \begin{cpp}
class Discounted_item : public Item {
  std::size_t min_quantity = 0;
  double discount = 1.0;
 public:
  Discounted_item(const std::string &s, double p,
                  std::size_t qty, double disc)
      : Item(s, p), min_quantity(qty), discount(disc) {}
  // other members
};
    \end{cpp}
    \begin{itemize}
        \item What if we don't call the ctor of the base class explicitly?
        \item Can we directly initialize the members of the base class?
        \begin{cpp}
Discounted_item(const std::string &s, double p,
                std::size_t qty, double disc)
    : name(s), price(p), min_quantity(qty),
      discount(disc) {}
        \end{cpp}
    \end{itemize}
\end{frame}

\begin{frame}[fragile]{Inheritance and Constructors}
    \begin{columns}
        \begin{column}{0.6\linewidth}
            Ctors are not automatically inherited, but we can inherit them explicitly:
            \begin{cpp}
class Binary_node {
 protected:
  Expr_node *lhs, *rhs;
  Binary_node(Expr_node *left,
      Expr_node *right)
      : lhs(left), rhs(right) {}
  // other members
};
class Plus_node
    : public Binary_node {
  using Binary_node::Binary_node;
  // other members
};
            \end{cpp}
        \end{column}
        \begin{column}{0.5\linewidth}
            then \ttt{Plus\_node} has a constructor
            \begin{cpp}
Plus_node(Expr_node *left, Expr_node *right)
  : Binary_node(left, right) {}
            \end{cpp}
            and we can call it by
            \begin{cpp}
Plus_node pn(a, b);
auto pnp
    = new Plus_node(a, b);
            \end{cpp}
        \end{column}
    \end{columns}
\end{frame}

\begin{frame}{Inheritance and Constructors}
    \begin{itemize}
        \item Default ctor and copy ctor won't be inherited by a \bluett{using} declaration. \red{(Why?)}
        \item All the ctors (except default ctor and copy ctor) are inherited by a \bluett{using} declaration. But the subclass can rewrite some.
        \begin{itemize}
            \item If the subclass has a ctor which has the same parameters as one of the ctors of the base class, then this ctor is hiding the corresponding one of the base class.
        \end{itemize}
        \item The access-level will be preserved. \red{(Why?)}
        \item The \bluett{explicit} attribute, if any, is also preserved.
        \item How will the inherited ctors initialize the members of the subclass?
    \end{itemize}
\end{frame}

\begin{frame}{Inheritance and \bluett{friend}s}
    Friendship cannot be inherited.
    \begin{itemize}
        \item Are you getting along well with your father's friends?
    \end{itemize}
\end{frame}

\begin{frame}{Inheritance and Copy-control}
    We will talk about this later...
\end{frame}

\subsection{Dynamic Binding}

\begin{frame}[fragile]{Upcasting}
    A reference or pointer to base class can be bound to an object of subclass. \red{(Why?)}
    \begin{cpp}
Discounted_item di = some_value();
Item &ir = di;  // Treat di as an Item object
Item *ip = &di;
    \end{cpp}
    But on such references or pointers, only the members of base class are accessible. \red{(Why?)}
\end{frame}

\begin{frame}[fragile]{Upcasting: Example}
    \begin{cpp}
inline void print_info(const Item &item) {
  std::cout << "Name: " << item.get_name()
            << ", price: " << item.net_price(1)
            << std::endl;
}
// in main
Discounted_item di = some_value();
Item i = some_other_value();
print_info(di);
print_info(i);
    \end{cpp}
\end{frame}

\begin{frame}[fragile]{Static Type and Dynamic Type}
    \begin{itemize}
        \item \textbf{static type} of an expression: The type known at compile-time.
        \item \textbf{dynamic type} of an expression: The real type of the object that the expression or variable is representing. \textbf{Known at runtime}.
    \end{itemize}
    \begin{cpp}
Discounted_item di = some_value();
Item &ir = di; // ir has static type Item &,
               // but dynamic type Discounted_item.
    \end{cpp}
\end{frame}

\begin{frame}[fragile]{Static Type and Dynamic Type}
    \begin{cpp}
inline void print_info(const Item &item) {
  std::cout << "Name: " << item.get_name()
            << ", price: " << item.net_price(1)
            << std::endl;
}
    \end{cpp}
    The static type of \ttt{item} is \const\ttt{Item \&}, but the dynamic type is unknown.
\end{frame}

\begin{frame}[fragile]{\virtual Functions}
    \begin{cpp}
inline void print_info(const Item &item) {
  std::cout << "Name: " << item.get_name()
            << ", price: " << item.net_price(1)
            << std::endl;
}
    \end{cpp}
    Which \ttt{net\_price} is called?
\end{frame}

\begin{frame}[fragile]{\virtual Functions}
    \begin{cpp}
class Item {
 public:
  virtual double net_price(std::size_t n) const;
  // other members
};
class Discounted_item : public Item {
 public:
  virtual double net_price(std::size_t n) const override;
  // other members
};
    \end{cpp}
\end{frame}

\begin{frame}{\virtual Functions}
    \begin{itemize}
        \item The dynamic type of parameter \ttt{item} is runtime-determined.
        \item Since \ttt{net\_price} is a \virtual function, which one is called is determined at \textbf{runtime}, so that the correct version is called.
        \item \textbf{late-binding}, or \textbf{dynamic-binding}.
    \end{itemize}
\end{frame}

\begin{frame}{Overriding a \virtual Function}
    To \override a \virtual function,
    \begin{itemize}
        \item The function must have parameters the same as the function in the base class has.
        \item The return-type of the function should be either \textbf{identical to} or \textbf{covariant with} \red{(What's this?)} that of the corresponding function in the base class.
        \item Don't forget the \const qualifier!
    \end{itemize}
    To make sure that your function overrides the one in the base class, use the \override keyword.
\end{frame}

\begin{frame}{Overriding a \virtual Function}
    \begin{itemize}
        \item An overriding function is still \bluett{virtual}, even if not explicitly declared.
        \item The best practice is to explicitly write `\bluett{virtual}' and `\bluett{override}'.
        \begin{itemize}
            \item The \override keyword lets the compiler check and report if the function is not actually overriding.
        \end{itemize}
        \item Distinguish between \textbf{overriding}, \textbf{overloading} and `\textbf{hiding}'.
        \begin{itemize}
            \item Avoid confusing cases in your program! Don't invite troubles for yourself.
        \end{itemize}
    \end{itemize}
\end{frame}

\begin{frame}[fragile]{\virtual Destructors}
    \begin{cpp}
Base *bp = some_value();
delete bp;
    \end{cpp}
    which destructor should be called by `\bluett{delete }\ttt{bp}'?
    \pause
    \begin{itemize}
        \item To make dynamic binding work correctly, the destructors must be \bluett{virtual}!
        \item The synthesized destructor is \textbf{non-virtual}, but we can:
        \begin{cpp}
virtual ~Base() = default;
        \end{cpp}
        \item If the dtor of the base class is \bluett{virtual}, the synthesized destructor is also \bluett{virtual}.
    \end{itemize}
\end{frame}

\begin{frame}[fragile]{Inheritance and Copy-control}
    Remember to copy the base part correctly! One possible way:
    \begin{cpp}
class Derived : public Base {
 public:
  Derived(const Derived &d)
    : Base(d), /* members of Derived */ {}
  Derived &operator=(const Derived &d) {
    Base::operator=(d);
    // copy members of Derived
    return *this;
  }
};
    \end{cpp}
\end{frame}

\begin{frame}{Synthesized Copy-control Functions}
    \begin{itemize}
        \item When will the compiler synthesize a copy-control function?
        \item What's the behavior of them?
        \item When will the compiler mark them as \blue{deleted}?
        \item What about default ctors?
    \end{itemize}
\end{frame}

\begin{frame}[fragile]{Slicing}
    Suppose \ttt{Base} and \ttt{Derived} have a \virtual function \ttt{foo}.
    \begin{cpp}
Derived d = some_value();
Base b = d;
b.foo();    // Base::foo or Derived::foo?
    \end{cpp}
    When using an object of a subclass to initialize or assign to an object of the base class, the copy-ctor or copy-assignment operator \textbf{of the base class} is called.
    \begin{itemize}
        \item Therefore, the sub-part of the object is ignored, or \textbf{sliced down}.
        \item Dynamic binding won't happen.
    \end{itemize}
\end{frame}

\begin{frame}[fragile]{Downcasting}
    \begin{cpp}
Base *bp = new Derived{};
    \end{cpp}
    We cannot access the members of the subclass through a pointer to the base class. We need a \textbf{downcasting}.
    \begin{itemize}
        \item As long as the following conditions are satisfied, you can make a downcasting:
        \begin{itemize}
            \item The pointer or reference to the base class is \textbf{indeed} bound to an object of the subclass.
            \item The base class and the subclass are polymorphic, which means that there is at least one \virtual function.
        \end{itemize}
        \item You can make a downcasting by \bluett{dynamic\_cast}:
        \begin{cpp}
Derived *dp = dynamic_cast<Derived *>(bp);
Derived &dr = dynamic_cast<Derived &>(*bp);
        \end{cpp}
    \end{itemize}
\end{frame}

\begin{frame}{Downcasting}
    \begin{itemize}
        \item \bluett{dynamic\_cast} may have a significant funtime cost.
        \item Several common ways to avoid \bluett{dynamic\_cast}, like writing a group of \virtual functions.
        \item \textit{Effective C++} Item 27 talks about type-casting.
        \item \textit{More Effective C++} Item 31 talks about some more complicated cases: Making functions virtual with respect to more than one object.
    \end{itemize}
    \begin{notice}
        Avoid \bluett{dynamic\_cast}, especially in performance-sensitive code.
    \end{notice}
\end{frame}

\subsection{Abstract Base Classes}

\begin{frame}[fragile]{Pure \virtual Functions}
    By defining a function to be \ttt{=0}, it is defined as a \textbf{pure virtual} function.
    \begin{itemize}
        \item A class with at least one pure virtual function is an \textbf{abstract class}.
        \item A pure virtual function can be overridden in a subclass. But if it is not overridden, the subclass is still abstract.
        \item Creating objects of a type that is an abstract class is \textbf{not allowed}.
    \end{itemize}
    \blue{Generally, virtual functions in the base class that do not have a reasonable behavior should be pure virtual, and such class should be abstract.}
\end{frame}

\begin{frame}{Example: Greedy Snake}
    \begin{columns}
        \begin{column}{0.5\linewidth}
            ``A super-speed game is a game.''
            \begin{figure}[h]
                \centering
                \includegraphics[scale=0.5]{figures/snake_original.png}
            \end{figure}
        \end{column}
        \begin{column}{0.5\linewidth}
            ``A classic-mode game is a game. A super-speed game is also a game.''
            \begin{figure}[h]
                \centering
                \includegraphics[width=\textwidth]{figures/snake_new.png}
            \end{figure}
        \end{column}
    \end{columns}
    It turns out that the super-speed mode has too many differences from the classic-mode, so I \textbf{refactored} the program according to the diagram on the right.
\end{frame}

\begin{frame}{Which One is Better?}
    \begin{columns}
        \begin{column}{0.5\linewidth}
            \begin{figure}[h]
                \centering
                \includegraphics[scale=0.6]{figures/shape_original.png}
            \end{figure}
        \end{column}
        \begin{column}{0.5\linewidth}
            \begin{figure}[h]
                \centering
                \includegraphics[scale=0.6]{figures/shape_new.png}
            \end{figure}
        \end{column}
    \end{columns}
\end{frame}

\begin{frame}{Which One is Better?}
    \begin{itemize}
        \item ``A square \textbf{is a} rectangle'' is correct, but sometimes this is deceptive. (\textit{Effective C++} Item 32, very important)
        \item The structure on the right can be extended easily: (\textbf{reusability})
        \begin{figure}[h]
            \centering
            \includegraphics[scale=0.6]{figures/shape_extended.png}
        \end{figure}
    \end{itemize}
\end{frame}

\begin{frame}[fragile]{A Pure \virtual Destructor}
    Sometimes a class should be abstract, but there seems to be no reasonable choice over which function should be pure virtual.
    \pause
    \begin{itemize}
        \item Define the destructor to be pure virtual, and provide another definition.
    \end{itemize}
    \begin{cpp}
class Base {
 public:
  virtual ~Base() = 0;
};
Base::~Base() {}
    \end{cpp}
    In fact, we can provide definitions for pure virtual functions.
\end{frame}

\begin{frame}[fragile]{More on Inheritance...}
    \begin{itemize}
        \item There is still one thing that is magic to us: the `\bluett{public}' keyword:
        \begin{cpp}
class Discounted_item : public Item {};
        \end{cpp}
        \item \public inheritance models `is-a', while \private inheritance models `is-implemented-in-terms-of'. What's that?
    \end{itemize}
\end{frame}

\section{Constant Expressions}

\end{document}